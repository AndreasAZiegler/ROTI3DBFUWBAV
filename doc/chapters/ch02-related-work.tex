% !TEX root = ../thesis.tex

% set counter to n-1:
\setcounter{chapter}{1}

\chapter{Related Work}

\section{Ultra-wideband (UWB)}
In this semester project, an \ac{UWB} system as described in \cite{Naegeli:2016} was used to get the location and the velocity of the object, relative to the measurement setup. This \ac{UWB} system is manually calibrated with the help of a motion capture system (VICON) and has an accuracy of $\approx 10 cm$ but also provides a distance in the z-direction in contrast to the used visual tracker described next, which only gives x- and y-coordinates in pixels. The adaption of the \ac{UWB} system used in this semester project is described in \autoref{ch:setup}.

\section{Vision based object tracking}
Because the currently best performing vision based object trackers are mostly built with \ac{KCF}, an implementation of the \ac{KCF} tracker \cite{henriques2015tracking} was used as vision based object tracker\cite{Haag:2015}. It implements the \ac{KCF} tracker, with FHOG features proposed in \cite{lsvm-pami} and a few adaption, more precisely speaking, the default scale adaption proposed in \cite{danelljan2014dsst}, a more robust filter update scheme from \cite{danelljan2014colorattributes} and a target loss functionality presented in \cite{bolme2010mosse}.

\ac{KCF} exploit the fact that translated and scaled patches, as normally used to train discriminative classifiers, are riddled with redundancies and therefore can be represented as a circulant matrix. As is standard with correlation filters, the input patches are weighted by a cosine window to smoothly remove discontinuities at the image boundaries, caused by the cyclic shifts. Circulant matrices can then be diagonalized with the Discrete Fourier Transform, which reduces storage as well as computation by several orders of magnitude. As demonstrated in \cite{henriques2015tracking} kernel regression has the same complexity as its linear counterpart with this approach.

Thanks to this properties of the circulant matrices, the \ac{KCF} tracker can be implemented with only a few lines of code. The bulk of the functionality of the \ac{KCF} tracker is implemented in three functions: "train" which implements \autoref{eq:train}, "detect" which implements \autoref{eq:detect}, and "kernel\_correlation" which implements \autoref{eq:kernelcorrelation}.

\begin{equation}\label{eq:train}
	\hat{\alpha} = \frac{\hat{y}}{\hat{\vec{k}}^{\vec x \vec x} + \lambda}
\end{equation}
where $\vec{k}^{\vec x \vec x}$ is the first row of the kernel matrix $\textbf{K}=C(\vec{k}^{\vec x \vec x})$, and $\hat{}$ denotes the DFT of a vector.

\begin{equation}\label{eq:detect}
	\hat{\vec f}(z) = \hat{\vec k}^{xz} \odot \hat{\vec \alpha}
\end{equation}
where $\vec k^{\vec x \vec z}$ is the kernel correlation of $\vec x$ and $\vec z$.

\begin{equation}\label{eq:kernelcorrelation}
	\vec{k}^{\vec x\vec x'} = \mathit{exp} \big( -\frac{1}{\sigma^2} (\left\Vert \vec x \right\Vert^2) + \left\Vert \vec x' \right\Vert^2 -2\mathcal{F}^{-1}(\sum_c \hat{\vec x}_c \odot \hat{\vec x'}_c) \big)
\end{equation}
For more detailed information and also the derivation of the presented equations, please see \cite{henriques2015tracking}.

As this method is much faster than other algorithms and can be implemented in only a few lines of code, it makes it really suitable to use on low-power devices as it is common in the robotics area.

\section{Extended Kalman Filter (EKF)}
The \acf{EKF} \cite{Chui2009} in this semester project is used to fuse the 3D position of the tracked object from the \ac{UWB} system and the 2D pixel coordinates from the vision based object tracker. The \ac{EKF} model and the update steps are described in detail in \autoref{ch:fusing}.